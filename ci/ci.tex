\secrel{Язык Си для встраиваемых систем}\secdown

Пока вы ждете доставки Arduino по почте, или решаете стоит ли ее покупать, мы
уже можем начать знакомиться с настоящим \term{системным программированием}. 

\secrel{Встраиваемая система}\label{embed}

Для обычного компьютера, мобильного телефона или веб-программирования сейчас
популярны и широко используются десятки компьютерных языков. В робототехникие и
цифровой электронике используется другой тип компьютеров, которые чаще всего
называют общим названием\ --- \term{встраиваемые системы}.

\begin{framed}
\noindent\term{Встраиваемой
системой}\ называют компьютер, который \emph{встраивается}\ в робота,
промышленное оборудование, систему управления автомобилем, центральный блок
управления умным домом и т.п. \emph{и работает внутри как часть устройства}.
\end{framed}

С такими встраиваемыми системами вы сталкиваетесь ежедневно дома\ --- это
домофон, стиральная машина, современный телевизор, электрическая плита или СВЧ
печь. Чаще всего нажимая кнопку разогрева или запуска стирки вы даже не
подозреваете, что пользуетесь в этот момент компьютером, а фактически\ ---
роботом.
\note{
Сейчас робототехника подается как что-то новое, инновационное и только входяшее
в нашу жизнь. На самом деле робот был создан и стал использоваться в
производстве еще в 1949 году, когда на обычный промышленный станок подставили
датчики положения и двигатели, и подключили к первым ЭВМ.\\
\url{https://vseochpu.ru/pervyj-stanok-s-chpu/}

Настоящий робот не будет похож на человека (разве что экзоскелет). Он должен
эффективно решать узкие задачи, поэтому с самого начала разработки вся его его
конструкция и ПО создается для выполнения этих задач. Робот не будет управляться
голосом или мозговыми волнами, пока эффективнее, надежнее и дешевле это делать с
кнопочного пульта. Он будет ездить на легко обслуживаемых гусеницах или колесах,
пока это будет возможно при выполнении практических задач. И робот никогда не
будет общаться с оператором голосом, потому что он не будет ждать пока робот
наговорится\ --- человеку проще и часто жизненно необходимо считывать десятки
параметров с монитора, и принимать решения в доли секунды.
}

\clearpage
Для программирования встраиваемых систем и \term{микроконтроллеров}\ всегда
используется только один язык: \emph{язык системного программирования
Си}\note{ANSI C89 или C11 без ООП, или специально ограниченный диалект
\Cpp, требующий программиста высокой квалификации, очень хорошо знающего работу
компиляторов ООП языков: как избежать использования динамической памяти,
обойтись без исключений, и не вылезти из ограничений аппаратуры и требований по
скорости и предсказуемости работы кода}.
Традиционно программисты делятся на два вида:
\begin{description}[nosep]
\item[прикладные программисты (верхнего уровня)]\ пишут код прикладных программ:
расчетные программы, офисные пакеты, игры и все программы, с
скороыми непосредственно общается пользователь
\item[системные программисты (нижнего уровня)]\ работают на уровне аппаратуры
компьютера, операционной системы, драйверов, базовых сетевых протоколов, а также
пишут средства разработки (компиляторы, отладчики и т.п.)
\end{description}

\secup
