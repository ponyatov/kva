\secrel{Язык Си для встраиваемых систем}\secdown

Пока вы ждете доставки Arduino по почте, или решаете стоит ли ее покупать, мы
уже можем начать знакомиться с настоящим \term{системным программированием}. 

\secrel{Встраиваемая система}\label{embed}

Для обычного компьютера, мобильного телефона или веб-программирования сейчас
популярны и широко используются десятки компьютерных языков. В робототехникие и
цифровой электронике используется другой тип компьютеров, которые чаще всего
называют общим названием\ --- \term{встраиваемые системы}.

\begin{framed}
\noindent\term{Встраиваемой
системой}\ называют компьютер, который \emph{встраивается}\ в робота,
промышленное оборудование, систему управления автомобилем, центральный блок
управления умным домом и т.п. \emph{и работает внутри как часть устройства}.
\end{framed}

С такими встраиваемыми системами вы сталкиваетесь ежедневно дома\ --- это
домофон, стиральная машина, современный телевизор, электрическая плита или СВЧ
печь. Чаще всего нажимая кнопку разогрева или запуска стирки вы даже не
подозреваете, что пользуетесь в этот момент компьютером.

Для программирования встраиваемых систем и \term{микроконтроллеров}\ всегда
используется только один язык: язык системного программирования Си. 

\secup
