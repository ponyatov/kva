\secrel{Язык Си для встраиваемых систем}\secdown

Пока вы ждете доставки Arduino по почте, или решаете стоит ли ее покупать, мы
уже можем начать знакомиться с настоящим \term{системным программированием}. 

\secrel{Встраиваемая система}\label{embed}\secdown

Для обычного компьютера, мобильного телефона или веб-программирования сейчас
популярны и широко используются десятки компьютерных языков. В робототехникие и
цифровой электронике используется другой тип компьютеров, которые чаще всего
называют общим названием\ --- \term{встраиваемые системы}.

\begin{framed}
\noindent\term{Встраиваемой
системой}\ называют компьютер, который \emph{встраивается}\ в робота,
промышленное оборудование, систему управления автомобилем, центральный блок
управления умным домом и т.п. \emph{и работает внутри как часть устройства}.
\end{framed}

С такими встраиваемыми системами вы сталкиваетесь ежедневно дома\ --- это
домофон, стиральная машина, современный телевизор, электрическая плита или СВЧ
печь. Чаще всего нажимая кнопку разогрева или запуска стирки вы даже не
подозреваете, что пользуетесь в этот момент компьютером, а фактически\ ---
роботом.\note{
Сейчас робототехника подается как что-то новое, инновационное и только входяшее
в нашу жизнь. На самом деле робот был создан и стал использоваться в
производстве еще в 1949 году, когда на обычный промышленный станок подставили
датчики положения и двигатели, и подключили к первым ЭВМ.\\
\url{https://vseochpu.ru/pervyj-stanok-s-chpu/}

Настоящий робот не будет похож на человека (разве что экзоскелет). Он должен
эффективно решать узкие задачи, поэтому с самого начала разработки вся его его
конструкция и ПО создается для выполнения этих задач. Робот не будет управляться
голосом или мозговыми волнами, пока эффективнее, надежнее и дешевле это делать с
кнопочного пульта. Он будет ездить на легко обслуживаемых гусеницах или колесах,
пока это будет возможно при выполнении практических задач. И робот никогда не
будет общаться с оператором голосом, потому что нет времени ждать пока робот
наговорится\ --- человеку проще и часто жизненно необходимо считывать десятки
параметров с монитора, и принимать решения в доли секунды.
}
Встраиваемую систему легко распознать\ --- вы не видите ее как самостоятельный
компьютер. Даже если вы начнете рассматривать внутренности сломанного робота, вы
ее не сразу найдете: вам придется воспользоваться гуглом чтобы найти на печатных
платах микросхемы по их маркировке, и определить какая из них может быть
процессором, внешней памятью, или блоком ввода сигналов.

Как самый легкодоступный пример\note{несложно найти сломанный HDD или дисковод
компакт-дисков, и рассмотреть маркировку микросхем под хорошей лупой}\ можно
привести плату жесткого диска:
сам по себе он является встраиваемой системой, которая входит в состав большого
персонального компьютера или сервера. У жесткого диска есть свой собственный
специализированный процессор, со своей памятью программ и данных, который
общается с большим компьютером по кабелю (интерфейс SATA, IDE, USB,..),
самостоятельно управляет двигателем пакета дисков, перемещает магнитные головки,
отрабатывает команды на чтение и запись, и (де)кодирует сигналы при записи. 

\clearpage
С другой встраиваемой системой вы тоже уже можете быть хорошо знакомы\ --- это
плата контроллера из конструктора Lego NXT:\\
\fig{height=0.6\textheight}{img/lego/nxtblock.jpg}
\fig{height=0.6\textheight}{img/lego/nxtpcb.jpg}\\
вы думаете что это компьютер робота, на самом деле всю работу делает вон та
мелкая микросхема в левом нижнем углу, и ее нельзя запрограммировать на языках
EV3 и в графических рисовалках, которыми вы пользовались на занятиях.

У встраиваемых систем есть некоторая ключевая особенность: у них очень
ограниченный набор \term{устройств ввода/вывода}\ с которыми непосредственно
взаимодействует пользователь\note{ничего сложнее набора кнопок, ``крутилок'' и
примитивного экранчика на пару строк текста, иногда графического и даже
сенсорного (чувствует нажатия)}. Очень часто такого ввода/вывода вообще нет,
\emph{встраиваемая система не имеет пользовательского интерфейса}, и общается
только с другими компьютерами, и элементами аппаратуры.

Другая особенность: в принципе не предусматривается установка дополнительных
программ, пользователю обычно недоступно ничего сложнее обновления прошивки
(целиком одним файлом). Иногда пользователю дается возможность
перепрограммирования, как в Лего или контроллерах для систем промышленной
автоматики, но программа пользователя и \term{микропрограмма}\ самой
встраиваемой системы всегда отделены друг от друга.

\clearpage
Для программирования встраиваемых систем и \term{микроконтроллеров}\ всегда
используется только один язык: \emph{язык системного программирования
Си}\note{ANSI C89 или C11 без ООП, или специально ограниченный диалект
\Cpp, требующий программиста высокой квалификации, очень хорошо знающего работу
компиляторов ООП языков: как избежать использования динамической памяти,
обойтись без исключений, и не вылезти из ограничений аппаратуры и требований по
скорости и предсказуемости работы кода}.
Традиционно программисты делятся на два вида:
\begin{description}[nosep]
\item[прикладные программисты (верхнего уровня)]\ пишут код прикладных программ:
расчетные программы, офисные пакеты, игры и все программы, с
которыми непосредственно общается пользователь
\item[системные программисты (нижнего уровня)]\ работают на уровне аппаратуры
компьютера, операционной системы, драйверов, обеспечивают работу сетевых
программ, баз данных, а также пишут средства разработки (компиляторы, отладчики
и т.п.)
\end{description}

\secrel{Особенности встраиваемых систем}

Встраиваемая система сильно ограничивает программиста по сравнению с обычным
компьютером, заставляя каждого разработчика быть не только системным
программистом, но обязательно хорошо разбираться в электронике:
\begin{description}
\item[очень маленький объем памяти]\ \\типовой объем ОЗУ от 20 К (\emph{кило}!)
байт
\end{description}

\secup

\secrel{Базовый Си}

\secup
