\clearpage\secly{Добро пожаловать!}\secdown

\noindent
Это книга для тех, кто уже вырос из наборов Lego Mindstorms.
Они хороши в самом начале освоения робототехники, но имеют большие
ограничения:

\begin{description}

\item[закрытая система, крайне ограниченный набор модулей]\ \\ты можешь писать
достаточно сложные программы, но набор базовых команд не может быть расширен,
твоя программа может быть выполнена только на процессорном модуле Lego, и нет
способов увеличить объем пемяти или добавить новые типы датчиков

\item[большая цена]\ \\не каждая семья может себе позволить отдать сумму,
равную цене простого ноутбука, если ты захочешь иметь комплект Lego дома

\end{description}
\clearpage

Самая главная проблема Lego: какой бы сложной не была ваша программа, какие бы
продвинутые алгоритмы вы не использовали, в реальном мире ваш робот или
устройство так и \textit{останется игрушкой}.

\fig{height=0.48\textheight}{img/lego/loom.jpg}
\fig{height=0.48\textheight}{img/lego/lathe.jpg}
\fig{height=0.48\textheight}{img/lego/drill.jpg}

Если нужно что-то более сложное, например создать устройство, выполняющее
практически полезную функцию, придется сделать над собой небольшое усилие, и
освоить те вещи, для которых конструкторы Lego предлагали готовые решения (или
наборот вовсе их не касались).

Отходя от готового комплекта аппаратуры, инструментария и сообщества
пользователей продукта, возникает множество вопросов, на большую часть которых
пытается дать ответы эта книга. Это не только электроника и программное
обеспечение\note{в том числе и инструментарий, типа компиляторов языков
программирования}, но и ``железо'' не в переносном а в изначальном смысле\ ---
механика, конструкция корпуса и элементов выполняющих нужную функцию,
электропривод и т.п. В эпоху торгашей все же не стоит забывать и о технологиях
самостоятельного изготовления \ref{diy}, удачно дополняющих покупные компоненты
\ref{ali}.

\secup
