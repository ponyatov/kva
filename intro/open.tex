
\secly{OpenSource + OpenHardware}

\fig{height=0.35\textheight}{img/OpenHardware.png}
\fig{height=0.35\textheight}{img/LinuxPowered.png}
\fig{height=0.35\textheight}{img/CHBZ.png}
\bigskip

\noindent
Эта книга написана согласно идеологии OpenSource и принципам открытых научных
публикаций: любой желающий может не только свободно использовать информацию,
конструкции \ref{diy}\ и библиотеки \ref{libs}, но и принять участие в работе
над самой книгой. \licence


\bigskip
\url{https://github.com/ponyatov/kva}\\

\clearpage 
Книга делается в системе верстки \LaTeX\note{популярна для верстки научных
изданий с большим количеством сложных формул, также понравится программистам и
компьютерам: в случае когда вам нужно генерировать достаточно объемные отчеты и
другие \textit{документы созданные автоматически}}, которая работает в пакетном
режиме:
вы готовите содержание используя plain text файлы и язык разметки, затем
запускаете \verb|pdflatex|\ и получаете на выходе .pdf файл \ref{whyebook}. Весь
проект размещен на GitHub, и если вы готовы освоить язык разметки
\cite{lvovsky}\ и систему контроля версий Git \ref{git}, присылайте мне свои
pull requestы.

\medskip 
\url{https://github.com/ponyatov/kva/releases}

\bigskip
\emph{Если} вы нашли ошибки, или \emph{в тексте попалось слишком много
непонятных слов, обязательно присылайте мне свои замечания}\ --- мне нужна
обратная связь, чтобы сделать книгу без сюсюканья, но написанной в простом
стиле, понятном любому школьнику\note{образцом такого стиля явлются
книги Я.И.Перельмана\cite{perelman}}.

HEAD
\clearpage
\subsecly{Linux и открытое ПО}

В этой книге вы не найдете упоминаний ни одного коммерческого программного
продукта, требующего покупки для полноценного использования, или
стимулирующих нарушение законов об авторском праве. Единственное исключение
сделано для самизнаетекакойОС и 
\begin{framed}
\begin{center}
тут может быть ваша реклама
\end{center}
{\tiny * платежи принимаются в биткоинах и на анонимный счет в банке
Республики Гондурас **}\\
{\tiny ** буду рад, если найдется хоббит-электронщик из Тегусигальпа со знанием
английского или русского для перевода книги на испанский}
\end{framed}

\clearpage\noindent
СамизнаетекакаяОС уже куплена вашим работодателем\note{Стоимость базовой
лицензии чуть выше МРОТ, поэтому если ваш работодатель не предоставил вам
лицензионной копии, стоит задуматься стоит ли с ним сотрудничать.}
%\note{если нет, зачем вы работаете на это жлобьё?!}
или идет предустановленной на ноутбук сразу из магазина. Но в этом случае вы
имеете реальные шансы поиметь проблем с вирусами и \term{malware}: любая система
имеет ``дыры'' в безопасности, но ее массовое распространение умноженное на
низкую техничскую грамотность пользователей делает ее первой точкой атаки для
хакеров и злоумышленников (независимо от наличия самых крутых
антивирусов, сканеров безопасности и систем аудита).

Совсем другое дело прикладное ПО, особенно специфичное, типа САПР, универсальных
расчетных программ, систем разработки печатных плат и коммерческих компиляторов:
минимальный набор коммерческого ПО необходимого для работы фирмы в области
разработки оборудования и электроники, легко превысит пару миллионов рублей:

\bigskip\noindent
\begin{tabular}{l l l}
Microsoft Windows 10 & 12 716 руб. & Debian Linux x64 \\
M\$ Office Home and Business 2019 & 15 711 руб. & LibreOffice \\
КОМПАС-3D V18 & 146 000 руб. & FreeCAD \\
Dessault SOLIDWORKS Standard & от 374 000 р &\\
ANSYS Academic Research& 1947295 руб &\\
ANSYS Multiphisics & 6160021 руб &\\
DipTrace Full & 35000  & KiCAD \\
Eremex Delta Design float & 130000 руб &\\
\end{tabular}\\
{\small * минимальные цены коммерческих лицензий на февраль 2019 г.}


\clearpage

\subsecly{Почему выбрана лицензия CC BY-ND}\label{whynd}
*****
\clearpage

На самом деле стоимость этого ПО не такая уж и большая, по крайней мере оно
окупается (иначе оно бы не продавалось за такие суммы). Но для этого у вас уже
должен быть успешный раскрученный бизнес с десятками клиентов. Если вы только
решаете, стоит ли заниматься этой темой, или просто учитесь, это ПО просто не
для вас.

На самом начальном этапе вы вполне можете обойтись возможностями
бесплатного \textit{свободного} ПО, а если только открываете собственный бизнес\
--- обойтись только компьютером, без начального капитала на закупку ПО.

\clearpage
\subsecly{Почему выбрана лицензия CC BY-ND}
90398777e63692f7c63b4d41481aed88340a87d3

Wikipedia:
\href{https://ru.wikipedia.org/wiki/%D0%9B%D0%B8%D1%86%D0%B5%D0%BD%D0%B7%D0%B8%D0%B8_%D0%B8_%D0%B8%D0%BD%D1%81%D1%82%D1%80%D1%83%D0%BC%D0%B5%D0%BD%D1%82%D1%8B_Creative_Commons}{Лицензии и инструменты Creative Commons}
\bigskip

\begin{description}
\item[Creative Commons Attribution (сокращённо CC BY)]\ \\
Лицензия «С указанием авторства»\ --- при распространении копий книги, частей
библиотек и модулей \textbf{на них} или в сопроводительной документации
обязательно должно указываться авторство и ссылка на эту книгу.

\item[нет NC]\ \\
Я не против коммерческого распространения копий, и производства модулей для
продажи, но покупатель всегда должен иметь свободный и бесплатный доступ
к полной документации и \emph{исходным кодам прошивок}. Я не хочу чтобы
библиотеки имели проблему Linux, который подвержен массовой \term{тивоизации}:
купив большинство электроники, использующей в прошивке ядро Linux, практически
невозможно исправить или обновить прошивку или хотя бы версию ядра, так как
производитель не предоставляет доступа к исходному коду драйверов, и своим
правкам исходного кода ядра и другого программного обеспечения.\\
С другой стороны коммерческое использование предполагает выплату гонорара, и
оплату доработок книги и библиотек\ --- улучшающаяся документация и исправления
в дизайне указывают на наличие полноценной поддержки продукта, и стимулируют
спрос.

\item[BY-ND]\ (библиотеки могут быть использованы в составе устройств).\\
Чтобы книга развивалась, исправлялись ошибки, текст становился более понятным
любому школьнику, добавлялись новые интересные применения и полезные модули,
мне необходима обратная связь от читателей и пользователей. Появление
``форков'' книги и библиотек, которые делают другие авторы, приводит к размытию
сообщества между оригиналом и клонами, и ведет к потере обратной связи.

\medskip
К использонию библиотек в производных произведениях требования менее строгие\
--- вы можете использовать трассировку модулей или только их схему\note{чтобы
перетрассировать \emph{как часть своей печатной платы}}\ как часть вашего
устройства, но не в виде модуля-клона для продажи от своего имени, отличающегося
только разводкой, цветом печатной платы и парой разъемов. То же относится и к
исходному коду\ --- вы свободны модифицировать его под свои устройства, но мне
нужна обратная связь, чтобы знать что вас не устроило в оригинале, и не нужно ли
изменить оригинал чтобы он был более универсальным, и содержал меньше ошибок.

\end{description}

\clearpage
Также я не хочу участвовать в параде учебников по ``Одурине''\ --- вы легко
найдете в сети и книжном магазине десятки аналогичных учебников разных авторов,
сделанных по одному шаблону, содержащих одни и те же упражнения и примеры для
одних и тех же коммерческих наборов-конструкторов\note{имеющих стоимость в
десятки раз дороже тех же деталей купленных в ближайшем магазине электронных
компонентов или по почте \ref{ali}}. Или запустить аналогичный парад
клонопособий для инноваторов третьего года обучения.

Целевая аудитория книги\ --- скорее продвинунутые ARMатурщики \copyright\
которым оказалось тесно в рамках Одурины, и они хотят перейти на современные
микроконтроллеры из семейства ARM Cortex, практически применимые устройства, или
наоборот прикольные поделки сочетающие традиционные технологии с возможностями
современной цифровой электроники.
