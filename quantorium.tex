\clearpage\fig{width=\textwidth}{tmp/logo63.png}

\secly{Кванториум}

\noindent
Детские технопарки «Кванториум» – это площадки\note{Создаются за счет средств
федерального бюджета и внебюджетных источников во исполнение Поручения
Президента РФ Пр-1205 от 27 мая 2015 г.}, оснащенные высокотехнологичным
оборудованием, нацеленные на подготовку новых высококвалифицированных инженерных
кадров, разработку, тестирование и внедрение инновационных технологий и идей.

\clearpage

\begin{description}

\item[Миссия] содействовать ускоренному техническому развитию детей и реализации
научно-технического потенциала российской молодежи, внедряя эффективные модели
образования, доступные для тиражирования во всех регионах страны.

\item[Цель] создание и развитие системы современных инновационных площадок
интеллектуального развития и досуга для детей и подростков на территории России.

\bigskip
\url{https://vk.com/kvantorium63}

\clearpage
\item[Задачи]\ \\

\begin{itemize}[nosep]
  \item 
Создать систему научно-технического просвещения через привлечение детей и
молодёжи к изучению и практическому применению наукоёмких технологий.
  \item 
Выстроить социальный лифт для молодежи, проявившей значительные таланты в
научно-техническом творчестве.
  \item 
Обеспечить подготовку национально-ориентированного кадрового резерва для
наукоемких и высокотехнологичных отраслей экономики РФ.
  \item 
Разработать и внедрить новый российский формат дополнительного образования детей
в сфере инженерных наук.
  \item 
Обеспечить системное выявление и дальнейшее сопровождение одаренных в инженерных
науках детей.
\end{itemize}
 
\end{description}

\subsecly{Образовательные модули Кванториума}

Образовательная программа каждого направления «Кванториума» делится на модули по
возрастающей сложности. Обучение детей начинается с вводного модуля. Это самый
ответственный модуль, потому что от успеха или неудач при обучении на данном
модуле зависит заинтересуется учащийся наукой и изобретательством и будет дальше
учиться в «Кванториуме» или уйдет. Поэтому, модуль должен быть полезным,
формирующим практические навыки, и в то же время интересным; задачи, решаемые в
модуле, сложными, но в то же время достижимыми. При прохождении модуля у каждого
учащегося должна быть своя история успеха, которая создается через преодоление
трудностей. Создать ситуацию успеха, это значит помочь ученику перейти от «Как
это?» к «Я могу!»
