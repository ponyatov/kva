\begin{thebibliography}{99}
\addcontentsline{toc}{section}{Литература}

\clearpage

\bibitem{kernigan} \fig{height=0.5\textheight}{bib/kernigan.jpg}\\
	Брайан У. Керниган, Деннис М. Ритчи\\
	\href{https://www.ozon.ru/context/detail/id/2480925/}{\textbf{Язык
	программирования C}}
	
Классическая книга по языку С, написанная самими разработчиками этого языка и
выдержавшая в США уже 34 переиздания! Книга является как практически
исчерпывающим справочником, так и учебным пособием по самому распространенному
языку программирования. Для чтения требует знания основ программирования и ЭВМ.

\clearpage

\bibitem{deitel} \fig{height=0.5\textheight}{bib/deitel.jpg}\\
	Пол Дейтел, Харви Дейтел
	\href{https://www.ozon.ru/context/detail/id/24769512/}{Как программировать на
	С} (и \Cpp)
	
Книга является общепризнанным руководством для изучения языка (ISO) $C_{11}$,
который широко распространен на различных платформах, включая Windows и
UNIX/Linux. Кроме того, дополнительно излагается \Cpp\ --- язык, являющийся
логическим развитием \Ci\ в сторону современных методологий программирования,
таких, как объектно-ориентированное и обобщенное (шаблоны) программирование.

\clearpage

\bibitem{gololob} \fig{height=0.5\textheight}{bib/gololob.jpg}\\
	Гололобов В.Н.\\
	\href{https://drive.google.com/open?id=1ctDWg_BTIIfJmy11MSw2ktIN0wblXter}{С
	чего начинаются роботы. О проекте Arduino для школьников}
	
Очень подробный учебник по платформе Arduino, включает пошаговое руководство по
установке программ для Windows и Linux, примеры исходного кода, схемы и
рекомендации.

\clearpage

\bibitem{bachinin} \fig{height=0.5\textheight}{bib/bachinin.png}\\
	Бачинин А., Панкратов В., Накоряков В.\\
	\href{http://examen-technolab.ru/instuctions/tv-0441-m-1.pdf}{Основы программирования
	микроконтроллеров:\\учебное пособие к образовательному набору "Амперка"}
	
\clearpage

\bibitem{ermishin} \fig{height=0.5\textheight}{bib/ermishin.jpg}\\
	Ермишин Н. В., Панфилов А. О., Косаченко С. В.\\
	\href{http://examen-technolab.ru/instuctions/tv-0441-m-2.pdf}{Основы
	робототехники;\\образовательный робототехнический модуль (базовый уровень)}


\end{thebibliography}
