\begin{thebibliography}{99}
\addcontentsline{toc}{section}{Литература}

\clearpage

\bibitem{kernigan} \fig{height=0.5\textheight}{bib/kernigan.jpg}\\
	Брайан У. Керниган, Деннис М. Ритчи\\
	\href{https://www.ozon.ru/context/detail/id/2480925/}{\textbf{Язык
	программирования C}}
	
Классическая книга по языку С, написанная самими разработчиками этого языка и
выдержавшая в США уже 34 переиздания! Книга является как практически
исчерпывающим справочником, так и учебным пособием по самому распространенному
языку программирования. Для чтения требует знания основ программирования и ЭВМ.

\clearpage

\bibitem{deitel} \fig{height=0.5\textheight}{bib/deitel.jpg}\\
	Пол Дейтел, Харви Дейтел
	\href{https://www.ozon.ru/context/detail/id/24769512/}{Как программировать на
	С} (и \Cpp)
	
Книга является общепризнанным руководством для изучения языка (ISO) $C_{11}$,
который широко распространен на различных платформах, включая Windows и
UNIX/Linux. Кроме того, дополнительно излагается \Cpp\ --- язык, являющийся
логическим развитием \Ci\ в сторону современных методологий программирования,
таких, как объектно-ориентированное и обобщенное (шаблоны) программирование.

\end{thebibliography}
